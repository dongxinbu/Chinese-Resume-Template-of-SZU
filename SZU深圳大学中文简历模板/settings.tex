\documentclass[11pt]{article}
% 文档类设置为 article,字体大小为 11pt,可修改为 10pt 或 12pt 以调整整体字体大小。

% -------------------- 加载常用宏包 --------------------
% 只是使用模板的深大友友不需要管这部分,需要额外的宏包直接插入到这里面就行,正常不会冲突。
\usepackage{xltxtra}         % 提供 XeTeX 特性扩展,自动加载 fontspec 和 xunicode
\usepackage{bookmark}        % 生成 PDF 书签,便于导航
\usepackage{hyperref}        % 启用超链接支持
\hypersetup{hidelinks}       % 去掉超链接的边框,仅保留跳转功能

\usepackage{url}             % 美化 URL 的排版
\urlstyle{tt}                % 将 URL 显示为等宽字体(更适合代码或技术文档)

\usepackage{multicol}        % 支持多栏布局
\usepackage{xcolor}          % 提供颜色支持,用于字体、背景等
\usepackage{calc}            % 支持简单的算术计算,用于布局调整
\usepackage{graphicx}        % 提供插图支持
\usepackage{tikz}            % 矢量图绘制工具
\usetikzlibrary{calc}        % 扩展 tikz 功能,支持节点位置计算

\usepackage{fontspec}        % 设置字体,支持 TrueType 和 OpenType
\usepackage{xeCJK}           % 支持中英文字体设置及混排优化
\usepackage{relsize}         % 支持调整字体相对大小
\usepackage{xspace}          % 优化命令后空格处理
\usepackage{fontawesome5}    % 引入 FontAwesome 字体图标

\usepackage{titlesec}        % 自定义标题格式
\usepackage{enumitem}        % 优化列表格式
\usepackage{siunitx}         % 支持数值和单位的正确排版
\usepackage{amssymb}         % 提供额外数学符号支持
\usepackage{tabularx}        % 支持自适应宽度表格
\usepackage{fancybox}        % 创建盒子、阴影等效果
\usepackage{float}           % 控制浮动对象位置(如表格、图片)

% -------------------- 自定义设置 --------------------
% 这部分使用者不要改动,不然会出现奇怪的问题。
% (参考自@LeyuDame,网址:https://github.com/LeyuDame/BNUCV/tree/main)
% 1. 解决中文字符与数字间自动插入空格问题
\CJKsetecglue{}

% 2. 美化 C++ 的显示方式
\protected\def\Cpp{{C\nolinebreak[4]\hspace{-.05em}\raisebox{.28ex}{\relsize{-1}++}}\xspace}

% 3. 段落设置:取消段首缩进
\setlength{\parindent}{0pt}

% 4. 取消页码显示(可根据需要开启页码)
\pagenumbering{gobble}

% 5. 列表格式优化:调整顶部间距和左缩进
\setlist[itemize]{topsep=0em, leftmargin=*}
\setlist[enumerate]{topsep=0em, leftmargin=*}

% -------------------- 自定义标题样式 --------------------
% 标题可以根据你自己的需要按照注释调整
% 一级标题样式
\titleformat{\section}                       % 自定义 \section 格式
  {\LARGE\bfseries\raggedright}              % 设置字体为 LARG、加粗、左对齐
  {}{0em}                                    % 标题前无编号
  {}                                         % 标题格式结束
  [{\color{SZU_Pink}\titlerule}]             % 标题下方添加彩色横线(深大荔枝红)
\titlespacing*{\section}{0cm}{*1.2}{*1.2}    % 一级标题的前后间距设置

% 二级标题样式
\titleformat{\subsection}
  {\large\bfseries\raggedright}              % 设置字体为 large、加粗、左对齐
  {}{0em}                                    % 标题前无编号
  {}                                         % 标题格式结束
  []                                         % 无下划线
\titlespacing*{\subsection}{0cm}{*1.2}{*1.2} % 二级标题的前后间距设置

% -------------------- 页面布局设置 --------------------
% 这部分也是可以改动的,如果你觉得上下左右哪些地方留空白太多或者太少可以手动调整,下面注释很清晰不多说了。
\usepackage[
  a4paper,        % 页面大小
  left=1.2cm,     % 左边距
  right=1.2cm,    % 右边距
  top=1.5cm,      % 上边距
  bottom=1cm,     % 下边距
  nohead          % 取消页眉
]{geometry}

% -------------------- 行间距与表格行距设置 --------------------
% 同理按需改动
\renewcommand{\arraystretch}{1.2}  % 表格行高调整为 1.2 倍
\linespread{1.2}                   % 正文行距调整为 1.2 倍

% -------------------- 字体设置 --------------------
% 可以自行更换字体,注意放到fonts文件夹中,并且在以下的SweiSpring替换为你的字体名字,加粗命名规则不变。
% 设置英文字体
\setmainfont[
    Path=fonts/,     % 字体所在路径
    Extension=.ttf,  % 字体文件扩展名
    BoldFont=* Bold, % 加粗字体文件名规则
]{SweiSpring}

% 设置中文字体
\setCJKmainfont[
    Path=fonts/,     % 字体所在路径
    Extension=.ttf,  % 字体文件扩展名
    BoldFont=* Bold, % 加粗字体文件名规则
]{SweiSpring}

% -------------------- 颜色定义 --------------------
% 定义深大荔枝红(来自深大官网深圳大学标识:https://www.szu.edu.cn/xxgk/sdbs.htm)
% 注意logo和配色有修改都需要根据官网官方pdf的要求,最好不要发挥你的主观能动性,不然挨老师批了找我就查无此人哈。
\definecolor{SZU_Pink}{RGB}{149, 0, 64}