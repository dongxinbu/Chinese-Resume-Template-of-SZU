% 有任何反馈请看readme.tex
\documentclass[11pt]{article}
% 文档类设置为 article,字体大小为 11pt,可修改为 10pt 或 12pt 以调整整体字体大小。

% -------------------- 加载常用宏包 --------------------
% 只是使用模板的深大友友不需要管这部分,需要额外的宏包直接插入到这里面就行,正常不会冲突。
\usepackage{xltxtra}         % 提供 XeTeX 特性扩展,自动加载 fontspec 和 xunicode
\usepackage{bookmark}        % 生成 PDF 书签,便于导航
\usepackage{hyperref}        % 启用超链接支持
\hypersetup{hidelinks}       % 去掉超链接的边框,仅保留跳转功能

\usepackage{url}             % 美化 URL 的排版
\urlstyle{tt}                % 将 URL 显示为等宽字体(更适合代码或技术文档)

\usepackage{multicol}        % 支持多栏布局
\usepackage{xcolor}          % 提供颜色支持,用于字体、背景等
\usepackage{calc}            % 支持简单的算术计算,用于布局调整
\usepackage{graphicx}        % 提供插图支持
\usepackage{tikz}            % 矢量图绘制工具
\usetikzlibrary{calc}        % 扩展 tikz 功能,支持节点位置计算

\usepackage{fontspec}        % 设置字体,支持 TrueType 和 OpenType
\usepackage{xeCJK}           % 支持中英文字体设置及混排优化
\usepackage{relsize}         % 支持调整字体相对大小
\usepackage{xspace}          % 优化命令后空格处理
\usepackage{fontawesome5}    % 引入 FontAwesome 字体图标

\usepackage{titlesec}        % 自定义标题格式
\usepackage{enumitem}        % 优化列表格式
\usepackage{siunitx}         % 支持数值和单位的正确排版
\usepackage{amssymb}         % 提供额外数学符号支持
\usepackage{tabularx}        % 支持自适应宽度表格
\usepackage{fancybox}        % 创建盒子、阴影等效果
\usepackage{float}           % 控制浮动对象位置(如表格、图片)

% -------------------- 自定义设置 --------------------
% 这部分使用者不要改动,不然会出现奇怪的问题。
% (参考自@LeyuDame,网址:https://github.com/LeyuDame/BNUCV/tree/main)
% 1. 解决中文字符与数字间自动插入空格问题
\CJKsetecglue{}

% 2. 美化 C++ 的显示方式
\protected\def\Cpp{{C\nolinebreak[4]\hspace{-.05em}\raisebox{.28ex}{\relsize{-1}++}}\xspace}

% 3. 段落设置:取消段首缩进
\setlength{\parindent}{0pt}

% 4. 取消页码显示(可根据需要开启页码)
\pagenumbering{gobble}

% 5. 列表格式优化:调整顶部间距和左缩进
\setlist[itemize]{topsep=0em, leftmargin=*}
\setlist[enumerate]{topsep=0em, leftmargin=*}

% -------------------- 自定义标题样式 --------------------
% 标题可以根据你自己的需要按照注释调整
% 一级标题样式
\titleformat{\section}                       % 自定义 \section 格式
  {\LARGE\bfseries\raggedright}              % 设置字体为 LARG、加粗、左对齐
  {}{0em}                                    % 标题前无编号
  {}                                         % 标题格式结束
  [{\color{SZU_Pink}\titlerule}]             % 标题下方添加彩色横线(深大荔枝红)
\titlespacing*{\section}{0cm}{*1.2}{*1.2}    % 一级标题的前后间距设置

% 二级标题样式
\titleformat{\subsection}
  {\large\bfseries\raggedright}              % 设置字体为 large、加粗、左对齐
  {}{0em}                                    % 标题前无编号
  {}                                         % 标题格式结束
  []                                         % 无下划线
\titlespacing*{\subsection}{0cm}{*1.2}{*1.2} % 二级标题的前后间距设置

% -------------------- 页面布局设置 --------------------
% 这部分也是可以改动的,如果你觉得上下左右哪些地方留空白太多或者太少可以手动调整,下面注释很清晰不多说了。
\usepackage[
  a4paper,        % 页面大小
  left=1.2cm,     % 左边距
  right=1.2cm,    % 右边距
  top=1.5cm,      % 上边距
  bottom=1cm,     % 下边距
  nohead          % 取消页眉
]{geometry}

% -------------------- 行间距与表格行距设置 --------------------
% 同理按需改动
\renewcommand{\arraystretch}{1.2}  % 表格行高调整为 1.2 倍
\linespread{1.2}                   % 正文行距调整为 1.2 倍

% -------------------- 字体设置 --------------------
% 可以自行更换字体,注意放到fonts文件夹中,并且在以下的SweiSpring替换为你的字体名字,加粗命名规则不变。
% 设置英文字体
\setmainfont[
    Path=fonts/,     % 字体所在路径
    Extension=.ttf,  % 字体文件扩展名
    BoldFont=* Bold, % 加粗字体文件名规则
]{SweiSpring}

% 设置中文字体
\setCJKmainfont[
    Path=fonts/,     % 字体所在路径
    Extension=.ttf,  % 字体文件扩展名
    BoldFont=* Bold, % 加粗字体文件名规则
]{SweiSpring}

% -------------------- 颜色定义 --------------------
% 定义深大荔枝红(来自深大官网深圳大学标识:https://www.szu.edu.cn/xxgk/sdbs.htm)
% 注意logo和配色有修改都需要根据官网官方pdf的要求,最好不要发挥你的主观能动性,不然挨老师批了找我就查无此人哈。
\definecolor{SZU_Pink}{RGB}{149, 0, 64}
% 引入设置文档settings.tex(不需修改)
% 右上方展示你的照片:删除images文件夹中的avatar图片,上传你自己的照片改为avatar.png,其余的别改动。
\newcommand{\school}{计算机与软件学院}
% 改为自己的学院,学院英文名不长的可在中文名后加上,用“|”隔开。
\newcommand{\contact}{
    \scriptsize
    \textcolor{white}{
        \faEnvelope \quad \href{mailto:youremail@SZU.edu.com}{xxxx@xxxmail.com}
        % 更改为自己的邮箱,尽量使用深大学生邮箱显得专业 ๑乛◡乛๑ (别惦记你那qq邮箱了 (╯‵□′)╯︵┻━┻   )
        \hspace{4em}
        \faPhone \quad xxx-xxxx-xxxx
        % 填自己手机号,海外考虑前置+86
    }
}

\begin{document}

% 页眉设计
\begin{tikzpicture}[remember picture, overlay]
    \node[anchor=north, inner sep=0pt](header) at (current page.north){
        \includegraphics[width=\paperwidth]{images/header.png}
    };
    \node[anchor=west](school_logo) at (header.west){\hspace{0.5cm}};
    \node[anchor=east](school_name) at(header.east){
        \textcolor{white}{\textbf{\school}}\hspace{0.5cm}
    };
\end{tikzpicture}
\vspace{-3.5em}

% 页脚设计
\begin{tikzpicture}[remember picture, overlay]
    \node[anchor=south, inner sep=0pt](footer) at (current page.south){
        \includegraphics[width=\paperwidth]{images/footer.png}
    };
    \node[anchor=center] at(footer.center){\contact};
\end{tikzpicture}

% 页面背景水印,深大官方logo,不需改动,不想要水印整模块删除即可
\begin{tikzpicture}[remember picture, overlay]
    \node[opacity=0.05] at(current page.center){
        \includegraphics[width=0.7\paperwidth, keepaspectratio]{images/szu_logo_big.png}
    };
\end{tikzpicture}

% 个人信息部分
\begin{figure}[h]
    \begin{minipage}{0.82\textwidth}
        \section{\makebox[\widthof{\faAddressCard}][c]{\color{SZU_Pink}{\faAddressCard}}\quad 个人信息}
        \begin{tabularx}{\linewidth}{p{\widthof{出生日期:}}Xp{\widthof{政治面貌:}}X}
            姓\ \ \ \ \ \ \ \ 名: & 你的名字 & 
            性\ \ \ \ \ \ \ \ 别: & 你的性别  \\
            出生年月:              & 你的出生年月 & 
            政治面貌:              & 你的政治面貌 \\
            % 想多加个人信息的按照以上格式加就行。
        \end{tabularx}
    \end{minipage}
    \hspace{2em}
    \begin{minipage}{0.12\textwidth}   
    % 头像照片不需一定正方形,长方形也是自动插入。
    % 调整0.12可调整图片宽度,但需同步调整上文中的\begin{minipage}{0.82\textwidth}中的0.82需要一增一减
        \setlength{\fboxsep}{0pt}
        \doublebox{\includegraphics[width=\linewidth]{images/avatar.png}}
    \end{minipage}

\end{figure}
\vspace{-1em}

% 教育背景部分
\section{\makebox[\widthof{\faGraduationCap}][c]{\color{SZU_Pink}{\faGraduationCap}}\quad 教育背景}
\vspace{0.5em}
{\large \textbf{深圳大学}},本科 \hfill {广东,深圳} \\
{{\href{https://csse.szu.edu.cn/}{计算机与软件学院}}},专业:人工智能 \hfill {2024年9月-2024年9月} \\
\textbf{主修课程}:1、2、3、4\ 等。
% 照着改就行应该不难,注意把计算机院的官网换成你们自己学院的,特别的需要提交电子简历的深大友友别忘了。

\vspace{0.5em}
{\large \textbf{深圳大学}},硕士 \hfill {广东,深圳} \\
{{\href{https://csse.szu.edu.cn/}{计算机与软件学院}}},专业:人工智能 \hfill {2024年9月-2024年9月} \\
\textbf{主修课程}:1、2、3、4\ 等。

\vspace{0.5em}
{\large \textbf{深圳大学}},博士 \hfill {广东,深圳} \\
{{\href{https://csse.szu.edu.cn/}{计算机与软件学院}}},专业:人工智能 \hfill {2024年9月-2024年9月} \\
\textbf{主修课程}:1、2、3、4\ 等。

% 科研成果部分
% 我知道大家很多都像我一样啥也没有,就三招:小奖照填,稍微夸大,现搓项目(别说是我说的,找我就查无此人(,,#゚Д゚)   )
\section{\makebox[\widthof{\faGraduationCap}][c]{\color{SZU_Pink}{\faGraduationCap}}\quad 科研成果}
\vspace{0.5em}
This is One of Your Paper Published in Conference A. \\
\textbf{Mingzi Nide}, Daoshi Nide. \hfill 发表于 \textbf{Conference A}(CCF-A类会议)

\vspace{0.5em}
This is Another Paper. \\
\textbf{Mingzi Nide}, Shidi Nide, Daoshi Nide. \hfill 发表于 \textbf{Conference B} (CCF-A类会议)

% 项目与教学部分
% 比较自由,填点当助教的经历,参加讲座的经历,项目经历,都行。改动就看着改。
\section{\makebox[\widthof{\faChalkboardTeacher}][c]{\color{SZU_Pink}{\faChalkboardTeacher}}\quad 项目与教学}
\vspace{0.5em}
{\large{\textbf{项目名称}}} \hfill {横向/纵向项目-已完结/进行中}\\
\textbf{你在项目中扮演的角色} \hfill 2020年9月至2021年9月\\
项目简介。

\vspace{1em}
{\large{\textbf{某某主题讨论班}}},主讲/参与 \hfill {2020年夏季}\\
主要内容:内容1,内容2,内容3\ 等。

\vspace{1em}
{\large{\textbf{课程名称}}},助教 \hfill {2021年夏季}\\
主要内容:内容1,内容2,内容3\ 等。

% 技能部分
% 照着我的格式改就行,当然了,别学我吹牛  (//▽//)
\section{\makebox[\widthof{\faWrench}][c]{\color{SZU_Pink}{\faWrench}}\quad 技能}
\vspace{0.5em}
\begin{itemize}
    \item 英语:取得CET-4与CET-6证书;
    \item 德语:完成A1至B2完整学习阶段;
    \item 编程:Python, MATLAB, C, C++。
    \item 工业软件:能熟练运用CAD, CATIA等工业软件。
\end{itemize}

\end{document}

